\documentclass[12pt, a4paper, twoside, romanian]{teza-upb}
\setcounter{secnumdepth}{3}
\setcounter{tocdepth}{3}
\usepackage{babel}
\usepackage{graphicx}
\usepackage{amsthm}
\usepackage{amsfonts}
\usepackage{url}
\usepackage[
  bookmarksnumbered,
  bookmarks,
  bookmarksopen=true,
  pdftitle={Dizertatie},
  linktocpage,
  dvips]{hyperref}


\singlespacing
\newtheorem{defn}{Definiție}

\begin{document}

\author{Mihai Chereji}

\title{Romagna: o abordare modernă asupra uneltelor de analiză conceptuală a datelor}


\facultatea{Facultatea de Electronică, Telecomunicații și Tehnologia Informației}
\tiplucrare{licență}
\domeniu{Calculatoare și tehnologia informației}
\catedra{Telecomunicații}
\program{Ingineria informației}
\titlulobtinut{Inginer}
\director{Christian Săcarea} 

\submissionmonth{Iulie} 
\submissionyear{2014} 

\beforepreface
\listoffigures
\listoftables
\abbreviations{ 
  HA = Harap Alb\\
  PLL =  Păsări-Lăți-Lungilă\\
  Sp  = Spânul 
  
}
\preface{
  
As defined in \cite{wiki:xxx}


Big-data și machine learning sunt domenii foarte căutate în zilele noastre, devenind chiar “buzzwords”. Majoritatea se bazează pe algoritmi simplii, de adunare și prelucrare automată a datelor.
Analiza conceptuală formală oferă o alternativă...

\begin{itemize}
  \item Importanța explorării conceptelor
  \item Lacunele softului existent
  \item Intențiile aplicației
\end{itemize}
}
\afterpreface 


\chapter{Analiza conceptuală formală}
  Analiza conceptuală formală este o ramură a teoriei laticelor, cu aplicabilitate extinsă asupra domeniului analizei calitative a datelor de orice fel.
  \section{Concepte matematice de bază}
    Întrucât scopul lucrării este de a descrie aplicația practică, ne vom limita la a descrie conceptele de care avem nevoie pentru a înțelege domeniul.
    \subsection{Mulțimi ordonate, latice, latice complete}
    \begin{defn}
      O mulțime $M$ este ordonată dacă se poate aplica asupra sa o relație $R$ care îndeplinește următoarele condiții:
      \begin{description}
        \item [Reflexivitate] $xRx$
        \item [Antisimetrie] $xRy, x \neq y \Rightarrow yRx$ e fals
        \item [Tranzitivitate] $xRy, yRz \Rightarrow xRz$
      \end{description}
      $\forall x, y,z \in M$. Relația $R$ se numește o \emph{relație de ordine}.
    \end{defn}

    Cel mai simplu exemplu, intuitiv exemplu este mulțimea numerelor reale $ \mathbb{R}$, alături de relația $\le$. Notăm o mulțime ordonată cu relația $\le$ cu $(M, \le)$.

    \begin{defn}
      Un element $y$ al mulțimii $M$ este \emph{vecinul superior} al lui $x$ dacă $x < y$ și nu există nici un $z$ astfel încât $x < z < y$. În mod invers, $x$ este \emph{vecinul inferior} al lui $y$.
    \end{defn}

    Putem nota relația de vecinătate astfel: $x \prec y, y \succ x$.

    \begin{defn}
      Fie $(M, \le)$ o mulțime ordonată, și $N$ o submulțime a sa. Înțelegem prin \emph{limita inferioară} a mulțimii $N$ un element $i$ astfel încât $\forall a \in N, i \le a$. În mod invers, \emph{limita superioară} a mulțimii $s$ este definită prin $\forall a \in N, s \ge a$.
      Putem nota mulțimea tuturor limitelor inferioare a grupului $N$ cu $I$. Elementul cel mai mare din această mulțime este numit \emph{minorantul} mulțimii $N$. Invers, cel mai mic element din mulțimea limitelor superioare este numit \emph{majorantul} mulțimii $N$.
    \end{defn}

    Minorantul se poate nota cu $\wedge N$ sau $\inf N$ (de la \emph{infimum}), iar majorantul cu $\vee N$ sau $\sup N$ (de la supremum).

    \begin{defn}
      O mulțime ordonată $M$ este numită o \emph{latice} dacă $\forall x,y \in M, \exists x \vee y, \exists x \wedge y$. În alte cuvinte, o mulțime ordonată este o latice dacă pentru orice 2 elemente
      ale mulțimii există majorant și minorant. O latice este \emph{completă} dacă pentru orice submulțime a ei există majorant și minorant.
    \end{defn}

    Orice latice completă are un element de vârf, numit \emph{elementul unitate}, și un element de bază, numit \emph{elementul zero}.

    \begin{defn}
    \end{defn}
    \subsection{Context, concept, latice de concepte}
    \begin{itemize}
      \item Scurtă istorie
      \item Concepte de bază
      \item Diagrame Hasse
      \item Conexiune Galois
      \item Latice conceptuală
      \item Algoritmi folosiți la generarea și afișarea laticelor.
    \end{itemize}

  \section{Introducere}
  \section{Teorie}
  \section{Utilizări practice}

\chapter{Starea actuală}
  \section{Toscana}
  \section{Toscanaj}

\chapter{Romagna}
  \section{Structură}
    \subsection{CoffeeScript}
    \subsection{Ember.js}
    \subsection{d3.js}

  \section{Raționament}

    \subsection{Ușurință de utilizare}

    \subsection{Avantajele distribuirii aplicațiilor web}

  \section{Dezvoltare}

    \subsection{Proces}

    \subsection{Probleme întâmpinate}
      \subsubsection{MySQL versus sql.js} % (fold)
      \label{ssub:MySQL versus sql.js}

      % subsubsection MySQL versus sql.js (end)

\bibliographystyle{alpha}
\bibliography{referinte}
\appendix
\chapter {Sup}
\end{document}
