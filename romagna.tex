\documentclass[12pt, a4paper, twoside, romanian]{teza-upb}
\setcounter{secnumdepth}{3}
\setcounter{tocdepth}{3}
\usepackage{babel}
\usepackage{graphicx}
\usepackage{amsthm}
\usepackage{pbox}
\usepackage{amsfonts}
\usepackage{url}
\usepackage[
  bookmarksnumbered,
  bookmarks,
  bookmarksopen=true,
  pdftitle={Dizertatie},
  linktocpage,
  dvips]{hyperref}


\singlespacing
\newtheorem{defn}{Definiție}
\newtheorem{example}{Exemplu}

\begin{document}

\author{Mihai Chereji}

\title{Romagna: o abordare modernă asupra uneltelor de analiză conceptuală a datelor}


\facultatea{Facultatea de Electronică, Telecomunicații și Tehnologia Informației}
\tiplucrare{licență}
\domeniu{Calculatoare și tehnologia informației}
\catedra{Telecomunicații}
\program{Ingineria informației}
\titlulobtinut{Inginer}
\director{Christian Săcarea} 

\submissionmonth{Iulie} 
\submissionyear{2014} 

\beforepreface
\listoffigures
\listoftables
\abbreviations{ 
  HA = Harap Alb\\
  PLL =  Păsări-Lăți-Lungilă\\
  Sp  = Spânul 
  
}
\preface{
  
As defined in \cite{wiki:xxx}


Big-data și machine learning sunt domenii foarte căutate în zilele noastre, devenind chiar “buzzwords”. Majoritatea se bazează pe algoritmi simplii, de adunare și prelucrare automată a datelor.
Analiza conceptuală formală oferă o alternativă...

\begin{itemize}
  \item Importanța explorării conceptelor
  \item Lacunele softului existent
  \item Intențiile aplicației
\end{itemize}
}
\afterpreface 


\chapter{Analiza conceptuală formală}
  Analiza conceptuală formală este o metodă de sistematizare a datelor în \textbf{concepte}, definite la modul larg ca mulțimi de obiecte care împărtășesc anumite atribute sau proprietăți. Este o reintepretare a teoriei clasică a laticelor, dezvoltată în principal în anii '30, axată către partea practică. Conceptul a fost introdus în lucrarea seminală a lui Robert Wille din 1982 \cite{wille:1982}, iar termenul a fost introdus în 1984 de același autor. În ultimele decenii, domeniul a atras multe contribuții și și-a dovedit utilitatea în domenii cum ar fi analiza și vizualizarea datelor, managementul informației.
  \section{Concepte matematice de bază}
    Întrucât scopul lucrării este de a descrie aplicația practică, ne vom limita la a descrie conceptele de care avem nevoie pentru a înțelege domeniul.
    \subsection{Mulțimi ordonate, latice, latice complete}
    \begin{defn}
      O mulțime $M$ este ordonată dacă se poate aplica asupra sa o relație $R$ care îndeplinește următoarele condiții:
      \begin{description}
        \item [Reflexivitate] $xRx$
        \item [Antisimetrie] $xRy, x \neq y \Rightarrow yRx$ e fals
        \item [Tranzitivitate] $xRy, yRz \Rightarrow xRz$
      \end{description}
      $\forall x, y,z \in M$. Relația $R$ se numește o \textbf{relație de ordine}.
    \end{defn}

    Cel mai simplu exemplu, intuitiv exemplu este mulțimea numerelor reale $ \mathbb{R}$, alături de relația $\le$. Notăm o mulțime ordonată cu relația $\le$ cu $(M, \le)$.

    \begin{defn}
      Un element $y$ al mulțimii $M$ este \textbf{vecinul superior} al lui $x$ dacă $x < y$ și nu există nici un $z$ astfel încât $x < z < y$. În mod invers, $x$ este \textbf{vecinul inferior} al lui $y$.
    \end{defn}

    Putem nota relația de vecinătate astfel: $x \prec y, y \succ x$.

    \begin{defn}
      Două elemente ale unei mulțimi ordonate sunt \textbf{comparabile} dacă $x \le y$ sau $y \le x$ (adică relația $\le$ se aplică asupra lor). Altfel sunt \textbf{incomparabile}. Un \textbf{lanț} este o submulțime în care oricare două elemente sunt comparabile. Un \textbf{antilanț} este o submulțime în care oricare două elemente sunt incomparabile.
    \end{defn}

    \begin{defn}
      Fie $(M, \le)$ o mulțime ordonată, și $N$ o submulțime a sa. Înțelegem prin \textbf{limita inferioară} a mulțimii $N$ un element $i$ astfel încât $\forall a \in N, i \le a$. În mod invers, \textbf{limita superioară} a mulțimii $s$ este definită prin $\forall a \in N, s \ge a$.
      Putem nota mulțimea tuturor limitelor inferioare a grupului $N$ cu $I$. Elementul cel mai mare din această mulțime este numit \textbf{minorantul} mulțimii $N$. Invers, cel mai mic element din mulțimea limitelor superioare este numit \textbf{majorantul} mulțimii $N$.
    \end{defn}

    Minorantul se poate nota cu $\wedge N$ sau $\inf N$ (de la \textbf{infimum}), iar majorantul cu $\vee N$ sau $\sup N$ (de la supremum).

    \begin{defn}
      O mulțime ordonată $M$ este numită o \textbf{latice} dacă $\forall x,y \in M, \exists x \vee y, \exists x \wedge y$. În alte cuvinte, o mulțime ordonată este o latice dacă pentru orice 2 elemente
      ale mulțimii există majorant și minorant. O latice este \textbf{completă} dacă pentru orice submulțime a ei există majorant și minorant.
    \end{defn}

    Orice latice completă are un element superior, numit \textbf{elementul unitate}, și un element inferior, numit \textbf{elementul zero}.

    \begin{defn}
    \end{defn}

    \subsection{Context, concept, ierarhie de concepte}
    \begin{defn}
      În cadrul analizei conceptuale, un \textbf{context} $K = (G, M, I)$ este format din 2 mulțimi, $G$ și $M$, și o relație binară $I$ între acestea. Mulțimea $G$ reprezintă obiecte, iar $M$ atribute.
    \end{defn}

      Literele provin din limba germană, în care conceptele au fost descrise inițial, de la Gegenstände și MerKmale, respectiv. Relația $I$ e numită \textbf{relația de incidență}, iar $gIm$ poate fi citit ca ``obiectul $g$ este descris de atributul $m$'', sau ``atributul $m$ descrie obiectul $g$''.

      Preluăm următorul exemplu din \cite{Carpineto:2004:CDA:975252}, un context (foarte redus)al animalelor vertebrate.
      \begin{example}
        \begin{table}
          \begin{tabular}[c]{| c | c | c | c | c | c | c | c | c | c | c |}
            \hline
            \multicolumn{2}{|c|}{} &
            \parbox{1.2cm}{\centering respiră în apă\\(a)} &
            \parbox{1.2cm}{\centering zboară \\(b)}         &
            \parbox{1.2cm}{\centering  are cioc\\(c)}       &
            \parbox{1.2cm}{\centering  are mâini \\(d)}      &
            \parbox{1.2cm}{\centering  are schelet \\(e)}     &
            \parbox{1.2cm}{\centering are aripi \\(f)}       &
            \parbox{1.2cm}{\centering  trăieșe în apă\\(g)}  &
            \parbox{1.2cm}{\centering naște pui vii \\ (h)}  &
            \parbox{1.2cm}{\centering  produce lumină \\(i)} \\ \hline
              1 & Liliac        &   & x &   &   & x & x &   & x &     \\
              2 & Vultur        &   & x & x &   & x & x &   &   &     \\
              3 & Maimuță       &   &   &   & x & x &   &   & x &     \\
              4 & Pește papagal & x &   & x &   & x &   & x &   &     \\
              5 & Pinguin       &   &   & x &   & x & x & x &   &     \\
              6 & Rechin        & x &   &   &   & x &   & x &   &     \\
              7 & Pește lanternă& x &   &   &   & x &   & x &   &  x  \\
            \hline
            \end{tabular}
          \caption{Un context al animalelor vertebrate.}
          \end{table}
      \end{example}

    Pentru $A \subseteq G$, definim
    $ A' = \{m \in M | gIm, \forall g \in A\} $.

    În mod asemănător, pentru $B \subseteq M$, $B' = \{g \in G | gIm, \forall m \in B \}$.

    În cuvinte, $A'$ este mulțimea tuturor atributelor (din contextul la care ne raportăm) care descriu toate obiectele din $A$.

    \begin{defn}
      Un \textbf{concept} al contextului $(G, M, I)$ este definit de $A \subseteq G$, $B \subseteq M$, unde $A' = B$ și $B' = A$.
    \end{defn}

    În engleză, mulțimea $A$ (a tuturor obiecte descrise de atributele conceptului) este numită \textbf{extent}, iar $B$ (atributele care descriu toate obiectele conceptului) \textbf{intent}.

    Având în vedere că $ A': G \rightarrow M$ și $B' : M \rightarrow G$, cei doi operatori pot fi combinați pentru a crea $A''$ și $G''$, care au ca domeniu mulțimea submulțimilor $G$ și $M$ respectiv.

    \begin{itemize}
      \item Diagrame Hasse
      \item Conexiune Galois
      \item Latice conceptuală
      \item Algoritmi folosiți la generarea și afișarea laticelor.
    \end{itemize}

  \section{Introducere}
  \section{Teorie}
  \section{Utilizări practice}

\chapter{Starea actuală}
  \section{Toscana}
  \section{Toscanaj}

\chapter{Romagna}
  \section{Raționament}

    \subsection{Ușurință de utilizare}

    \subsection{Avantajele distribuirii aplicațiilor web}
  \section{Structură}
  \section{Tehnologii}
    \subsection{CoffeeScript}
    \subsection{Ember.js}
    \subsection{d3.js}
      \subsubsection{svg}
    \subsection{sql.js}
      \subsubsection{emscripten}


  \section{Dezvoltare}

    \subsection{Proces}

    \subsection{Probleme întâmpinate}
      \subsubsection{MySQL versus sql.js} % (fold)
      \label{ssub:MySQL versus sql.js}

      % subsubsection MySQL versus sql.js (end)

\bibliographystyle{alpha}
\bibliography{referinte}
\appendix
\chapter {Sup}
\end{document}
