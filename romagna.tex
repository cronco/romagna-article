\documentclass[12pt, a4paper, oneside, romanian]{teza-upb}
\setcounter{secnumdepth}{3}
\setcounter{tocdepth}{3}
\usepackage{babel}
\usepackage{graphicx}
\usepackage[
  bookmarksnumbered,
  bookmarks,
  bookmarksopen=true,
  pdftitle={Dizertatie},
  linktocpage,
  url,
  dvips]{hyperref}


\singlespacing

\begin{document}

\author{Mihai Chereji}

\title{Romagna: o abordare modernă asupra uneltelor de analiză conceptuală a datelor}


\facultatea{Facultatea de Electronică, Telecomunicații și Tehnologia Informației}
\tiplucrare{licență}
\domeniu{Calculatoare și tehnologia informației}
\catedra{Telecomunicații}
\program{Ingineria informației}
\titlulobtinut{Inginer}
\director{Christian Săcarea} 

\submissionmonth{Iulie} 
\submissionyear{2014} 

\beforepreface
\listoffigures
\listoftables
\abbreviations{ 
  HA = Harap Alb\\
  PLL =  Păsări-Lăți-Lungilă\\
  Sp  = Spânul 
  
}
%\preface{}
\afterpreface 
\chapter{Introducere}
Hello\cite{punguta1876}.


Big-data și machine learning sunt domenii foarte căutate în zilele noastre, devenind chiar “buzzwords”. Majoritatea se bazează pe algoritmi simplii, de adunare și prelucrare automată a datelor.
Analiza conceptuală formală oferă o alternativă…

\begin{itemize}
  \item Importanța explorării conceptelor
  \item Lacunele softului existent
  \item Intențiile aplicației
\end{itemize}

\chapter{Analiza conceptuală formală}
  Analiza conceptuală formală e... blabla
  \begin{itemize}
    \item Scurtă istorie
    \item Concepte de bază
    \item Diagrame Hasse
    \item Conexiune Galois
    \item Latice conceptuală
    \item Algoritmi folosiți la generarea și afișarea laticelor.
  \end{itemize}

  \section{Introducere}
  \section{Teorie}
  \section{Utilizări practice}

\chapter{Starea actuală}
  \section{Toscana}
  \section{Toscanaj}

\chapter{Romagna}
  \section{Structură}
    \subsection{CoffeeScript}
    \subsection{Ember.js}
    \subsection{d3.js}

  \section{Raționament}

    \subsection{Ușurință de utilizare}

    \subsection{Avantajele distribuirii aplicațiilor web}

  \section{Dezvoltare}

    \subsection{Proces}

    \subsection{Probleme întâmpinate}
      \subsubsection{MySQL versus sql.js} % (fold)
      \label{ssub:MySQL versus sql.js}

      % subsubsection MySQL versus sql.js (end)

\bibliographystyle{unsrt}
\bibliography{referinte}
\appendix
\chapter {Sup}
\end{document}
